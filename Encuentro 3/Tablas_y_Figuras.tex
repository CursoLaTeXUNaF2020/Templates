\documentclass[11pt]{beamer}
\usepackage[utf8]{inputenc}
\usepackage[spanish]{babel}
\usepackage[T1]{fontenc}
\usepackage{lmodern}
\usepackage{amsmath}
\usepackage{amscd}
\usepackage{amsfonts}
\usepackage{amssymb}
\usepackage{mathtools}
\usepackage{amsthm}
\usepackage{float}
\usepackage[scriptsize]{subfigure}
\usepackage{caption}
\setbeamertemplate{caption}[numbered]
\usepackage{tikz}
\usepackage{graphicx}
\usepackage{eso-pic}
\usepackage{multicol}
\usepackage{multirow}
\usetheme{Madrid}

\theoremstyle{plain} % default
\newtheorem{thm}{Teorema:}
\newtheorem{prop}{Proposición:}
\newtheorem{lem}{Lema:}
\newtheorem{pro}{Demostración:}

\theoremstyle{definition}
\newtheorem{defi}{Definición:}
\newtheorem{exa}{Ejemplo:}


\begin{document}
	%\author{Ferreira, Juan David }
	\title{Figuras y Tablas}
	%\subtitle{}
	%\logo{}
	%\institute{}
	%\date{}
	%\subject{}
	%\setbeamercovered{transparent}
	%\setbeamertemplate{navigation symbols}{}
	\begin{frame}[plain]
		\maketitle
	\end{frame}

    \begin{frame}
    	\frametitle{Contenidos}
    	\tableofcontents[pausesections]
    \end{frame}
    \section[Introducción]{Introducción.}
    \section[Insertando tablas y figuras]{Insertando tablas y figuras.}
    \section[Objetos flotantes]{Objetos flotantes.}
    \section[Color en tablas]{Color en tablas.}
    \section[Rotación de texto en celdas]{Rotación de texto en celdas.}
    \section[Ancho de columnas]{Ancho de columnas.}
    \section[Compilando figuras en \LaTeX{}]{Compilando figuras en \LaTeX{}.}
     %%  ========================================================================================================================================================== %%
    \begin{frame}
    	\frametitle{Movimientos de Reidemeister}
    	Los movimientos de Reidemeister $\mathcal{R}_{1}$, $\mathcal{R}_{2}$ y $\mathcal{R}_{3}$ son de mucha utilidad para comprender ciertos estudios sobre isotopías de ambiente e isotopías regulares como relación de equivalencia (solamente) generadas por los movimientos $\mathcal{R}_{2}$ y $\mathcal{R}_{3}$.
    	\vspace{0.2cm}
    	\begin{figure}[H]
    		\centering
    		\begin{tikzpicture}[scale=0.35]
    		\node (a) at (-2, 1.5) {{\large $\mathcal{R}_{1}.$}};                                                                                           % rama derecha
    		\node[yshift=-1.4cm] (b) at (-2, 1.5) {{\large $\mathcal{R}_{2}.$}};                                                                                      % rama derecha
    		\node[yshift=-2.8cm] (c) at (-2, 1.5) {{\large $\mathcal{R}_{3}.$}};                                                                                      % rama derecha
    		\draw[white,white,double=black,ultra thick,double distance=1.25pt] (0,3)      .. controls (1.5,3)    and (2.5,3)    ..                           %
    		(2.5,1.5)  .. controls (2.5, 0.5) and (1.5, 0.5) ..                           %
    		(1.5, 1.5) .. controls (1.5, 3)   and (3.5, 3)   ..  (4,3);                   % rama derecha
    		\draw[white,white,double=black,ultra thick,double distance=1.25pt] (1.5, 1.5) .. controls (1.5, 3) and (3.5, 3) .. (4,3);                        % rama derecha
    		%%   ==========================================================================================================================================  %%%%%%%%%%%%%%%%%%%
    		\node[xshift=1.7cm] (d) at (0, 1.5) {{\large $=$}};
    		\node[xshift=7cm] (d) at (0, 1.5) {{\large $;$}};                                                                                                % 
    		\draw[xshift=7cm,white,double=black,thick,double distance=1pt] (0,1.75) .. controls (1,2.5) and (3,2.75) .. (4,2);                               %
    		%%   ==========================================================================================================================================  %%%%%%%%%%%%%%%%%%%
    		\node[xshift=4.5cm] (e) at (0, 1.5) {{\large $=$}};                                                                                              %
    		\draw[xshift=14cm,white,white,double=black,ultra thick,double distance=1.25pt] (0,3)      .. controls (1.5,3)    and (2.5,3)    ..               % rama derecha
    		(2.5,1.5)  .. controls (2.5, 0.5) and (1.5, 0.5) ..               % rama derecha
    		(1.5, 1.5) .. controls (1.5, 3)   and (3.5, 3)   .. (4,3);        % rama derecha
    		\draw[xshift=14cm,white,white,double=black,ultra thick,double distance=1.25pt] (0,3) .. controls (1.5,3) and (2.5,3) .. (2.5,1.5);               % 
    		%%   ==========================================================================================================================================  %%%%%%%%%%%%%%%%%%%
    		\draw[yshift=-4cm,white,double=black,ultra thick,double distance=1pt] (3,0) .. controls (0,1.5) .. (3,3);                                        % rama derecha
    		\draw[yshift=-4cm,white,double=black,ultra thick,double distance=1pt] (0,0) .. controls (3,1.5) .. (0,3);                                        % rama izquierda
    		\node[xshift=1.7cm,yshift=-1.4cm] (f) at (0, 1.5) {{\large $=$}}; 
    		\node[xshift=5cm,yshift=-1.4cm] (g) at (0, 1.5) {{\large $;$}};                                                                               %
    		\draw[xshift=10cm,yshift=-4cm,white,double=black,ultra thick,double distance=1pt] (3,0) .. controls (0,1.5) .. (3,3);                            % rama derecha
    		\draw[xshift=7cm,yshift=-4cm,white,double=black,ultra thick,double distance=1pt] (0,0) .. controls (3,1.5) .. (0,3);                             % rama izquierda
    		%%   ==========================================================================================================================================  %%%%%%%%%%%%%%%%%%%
    		\draw[yshift=-8cm,white,white,double=black,ultra thick,double distance=1pt] (0,1.75) .. controls (1,2.5) and (3,2.75) .. (4,2);                  % rama derecha
    		\draw[yshift=-8cm,white,white,double=black,ultra thick,double distance=1pt] (0,3) .. controls (1.75, 1.25) and (2.25, 0.75) .. (4,0);            % rama izquierda
    		\draw[yshift=-8cm,white,white,double=black,ultra thick,double distance=1pt] (0,0) .. controls (1.75, 1.25) and (2.25, 0.75) .. (4,3);            % 
    		\node[xshift=1.7cm,yshift=-2.8cm] (g) at (0, 1.5) {{\large $=$}};                                                                               %
    		\draw[xshift=6cm,yshift=-8.5cm,white,double=black,ultra thick,double distance=1pt] (0,1.25) .. controls (1,0.5) and (3,0.25) .. (4,1);           % rama derecha
    		\draw[xshift=6cm,yshift=-8cm,white,double=black,ultra thick,double distance=1pt] (0,3) .. controls (1.75, 1.25) and (2.25, 0.75) .. (4,0);       % rama izquierda
    		\draw[xshift=6cm,yshift=-8cm,white,double=black,ultra thick,double distance=1pt] (0,0) .. controls (1.75, 1.25) and (2.25, 0.75) .. (4,3);       % 
    		%%   ==========================================================================================================================================  %%%%%%%%%%%%%%%%%%%
    		\draw[xshift=14cm,yshift=-8cm,white,white,double=black,ultra thick,double distance=1pt] (0,1.75) .. controls (1,2.5) and (3,2.75) .. (4,2);      % rama derecha
    		\draw[xshift=14cm,yshift=-8cm,white,white,double=black,ultra thick,double distance=1pt] (0,0) .. controls (1.75, 1.25) and (2.25, 0.75) .. (4,3);%
    		\draw[xshift=14cm,yshift=-8cm,white,white,double=black,ultra thick,double distance=1pt] (0,3) .. controls (1.75, 1.25) and (2.25, 0.75) .. (4,0);% rama izquierda
    		\node[xshift=6.7cm,yshift=-2.8cm] (g) at (0, 1.5) {{\large $=$}};
    		\node[xshift=4.2cm,yshift=-2.8cm] (g) at (0, 1.5) {{\large $;$}};                                                                                %
    		\draw[xshift=20cm,yshift=-8.5cm,white,double=black,ultra thick,double distance=1pt] (0,1.25) .. controls (1,0.5) and (3,0.25) .. (4,1);          % rama derecha
    		\draw[xshift=20cm,yshift=-8cm,white,double=black,ultra thick,double distance=1pt] (0,0) .. controls (1.75, 1.25) and (2.25, 0.75) .. (4,3);      %
    		\draw[xshift=20cm,yshift=-8cm,white,double=black,ultra thick,double distance=1pt] (0,3) .. controls (1.75, 1.25) and (2.25, 0.75) .. (4,0);      % rama izquierda
    		\end{tikzpicture}
    		\caption{Movimientos de Reidemeister}	
    	\end{figure}
    \end{frame}
    %----------------------------------------------------------------- DEFINICIÓN -------------------------------------------------------------------------------- %
	
\end{document}