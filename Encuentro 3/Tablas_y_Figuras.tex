\documentclass[11pt]{beamer}
\usepackage[utf8]{inputenc}
\usepackage[spanish]{babel}
\usepackage[T1]{fontenc}
\usepackage{lmodern}
\usepackage{amsmath}
\usepackage{amscd}
\usepackage{amsfonts}
\usepackage{amssymb}
\usepackage{mathtools}
\usepackage{amsthm}
\usepackage{float}
\usepackage[scriptsize]{subfigure}
\usepackage{caption}
\setbeamertemplate{caption}[numbered]
\usepackage{tikz}
\usepackage{graphicx}
\usepackage{eso-pic}
\usepackage{multicol}
\usepackage{multirow}
\usepackage{verbatim}
\usetheme{Madrid}

\theoremstyle{plain} % default
\newtheorem{thm}{Teorema:}
\newtheorem{prop}{Proposición:}
\newtheorem{lem}{Lema:}
\newtheorem{pro}{Demostración:}

\theoremstyle{definition}
\newtheorem{defi}{Definición:}
\newtheorem{exa}{Ejemplo:}

\begin{document}
	%\author{Ferreira, Juan David }
	\title{Figuras y Tablas}
	%\subtitle{}
	%\logo{}
	%\institute{}
	%\date{}
	%\subject{}
	%\setbeamercovered{transparent}
	%\setbeamertemplate{navigation symbols}{}
	\begin{frame}[plain]
		\maketitle
	\end{frame}

    \begin{frame}
    	\frametitle{Contenidos}
    	\tableofcontents[pausesections]
    \end{frame}
    \section[Introducción]{Introducción.}
    \section[Insertando tablas y figuras]{Insertando tablas y figuras.}
    %%  ========================================================================================================================================================== %%

    \begin{frame}[fragile]
    	\frametitle{Comandos o códigos de \LaTeX{}}
    	
    	El paquete \textcolor{red}{\texttt{tabular}} proporciona el entorno \textcolor{red}{\texttt{tabular}}, que puede calcular automáticamente el ancho de una columna para restringir una tabla dentro de un ancho horizontal preespecificado, independientemente de la longitud de las entradas en la tabla.
    	
    	\begin{figure}[!h]
    		\begin{minipage}[b]{0.25\textwidth}
    			\begin{block}{Tablas}
    				\begin{tabular}{|c|c|c|} \hline
    					$p$ & $q$ & $p \rightarrow q$ \\ \hline
    					0 & 0 & 1 \\
    					0 & 1 & 1 \\ \cline{1-2}
    					1 & 0 & 0 \\
    					1 & 1 & 1 \\ \hline
    				\end{tabular}
    			\end{block}
    		\end{minipage}
    		\hfill
    		\begin{minipage}[b]{0.7\textwidth}
    			\begin{verbatim}
    			\begin{tabular}{|c|c|c|}\hline
    			$p$ & $q$ & $p \rightarrow q$\\\hline
    			0 & 0 & 1 \\
    			0 & 1 & 1 \\\cline{1-2}
    			1 & 0 & 0 \\
    			1 & 1 & 1 \\\hline
    			\end{tabular}
    			\end{verbatim}
    		\end{minipage}
    	\end{figure}
    
        En este entorno, las opciones \textcolor{red}{\texttt{l}} (izquierda), \textcolor{red}{\texttt{c}} (centro), \textcolor{red}{\texttt{r}} (derecha) definen el posicionamiento vertical de las columnas. Las tablas se editan las líneas \textcolor{blue}{verticales} y \textcolor{red}{horizontales}. El modo matemático (que veremos más adelante) debe especificarse en una tabla.
    \end{frame}
    %%  ========================================================================================================================================================== %%
    \begin{frame}[fragile]
    	\frametitle{Tablas}
    	\begin{block}{¿Cómo manejamos las líneas en las Tablas?}
    		\begin{itemize}
    			\item Para agregar líneas \textcolor{blue}{verticales} se ponen marcas como | o || en la parte que corresponde al alineamiento de columnas.
    			\item Para agregar líneas \textcolor{red}{horizontales}, al final de cada fila se especifica
    			\begin{itemize}
    				\item \verb|\hline|: línea tan larga como la tabla
    				\item\verb|\cline{i-j}|: línea de columna \textit{i} a columna \textit{j}
    			\end{itemize}
    		\end{itemize}
    		En el entorno tabular discutido en los slides anteriores, se genera una columna mediante una de las opciones de  \textcolor{red}{\texttt{l}}, \textcolor{red}{\texttt{c}} y \textcolor{red}{\texttt{r}}. El ancho de una columna bajo cualquiera de estas opciones se hace igual a la longitud de la entrada más larga en esa columna. Esto puede extender una tabla incluso más allá del ancho de una página si la tabla tiene algunas entradas muy largas.
    	\end{block}
    \end{frame}
    %%  ========================================================================================================================================================== %%
    \begin{frame}[fragile]
    	\frametitle{Textos oblicuos (rotados) en tablas}
    	Si una tabla contiene algunas entradas largas, se puede guardar espacio imprimiendo dichas entradas en dirección vertical a través del entorno \textcolor{red}{\texttt{sideways}} definido en el paquete \textcolor{red}{\texttt{rotating}}. Una aplicación del entorno \textcolor{red}{\texttt{sideways}} se muestra en la página siguiente.
    \end{frame}
    %%  ========================================================================================================================================================== %%
    \begin{frame}[fragile]
    	\frametitle{Fusión de filas y columnas de tablas}
    	
    	Al presentar diferentes tipos de información en una tabla, a menudo se requiere que algunas celdas se combinen en una sola. Los paquetes \textcolor{red}{\texttt{multicolum}}y \textcolor{red}{\texttt{multirow}} proporcionan los comandos {\color{blue}{\verb|\multicolumn{}{}{}|}} y {\color{blue}{\verb|\multirow{}{}{}|}} para fusionar dos o más columnas y filas, respectivamente.
    \end{frame}    
    \section[Objetos flotantes]{Objetos flotantes.}
    \section[Color en tablas]{Color en tablas.}
    \section[Rotación de texto en celdas]{Rotación de texto en celdas.}
    \section[Ancho de columnas]{Ancho de columnas.}
    \section[Compilando figuras en \LaTeX{}]{Compilando figuras en \LaTeX{}.}
     %%  ========================================================================================================================================================== %%
    %----------------------------------------------------------------- DEFINICIÓN -------------------------------------------------------------------------------- %
	
\end{document}